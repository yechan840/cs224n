\documentclass{amsart}

\usepackage{hyperref}

\theoremstyle{plain} 
\newtheorem{theorem} {Theorem} [section]
\newtheorem{corollary} [theorem] {Corollary}

\theoremstyle{remark}
\newtheorem{remark}{Remark} 

\numberwithin{equation}{section}

\newcommand {\R} {\mathbb{R}}
\newcommand{\N} {\mathbb{N}}
\newcommand{\Z} {\mathbb{Z}}

\DeclareMathOperator{\rank} {\mathrm{rank}} 
\DeclareMathOperator{\proj} {\mathrm{proj}}


\newcommand{\inner} [2] { \langle #1 , #2 \rangle }


\begin{document}


\title{Understanding Word2Vec}
\author{Yechan Lee}
\date{\today}

\maketitle

\begin{enumerate}
\item For some given $o, c$
\[ 
\begin{split}
& \mathbf{y} = [0, \cdots, 1  \text{ (c-th position)}, \cdots, 0] \\
& \mathbf{\hat y} = [P(O=1|C=c), P(O=2|C=c), \cdots ] \\ 
& J_{\text{naitve-softmax}}(v_c, o, U) = -\log P(O=o|C=c) \\
& = - \log \hat y_0 = -\sum_{w \in \text{Vocab}} y_w \log \hat y_w 
\end{split}
\]


\item Let D be the dimension of word vector, W be the number of words in vocabulary. Then, shape of $\mathbf v_c$ is $[D, 1]$, $\mathbf U$ is $[W, D]$, $\mathbf y, \mathbf {\hat y}$ is $[W, 1]$ 
\[
\begin{split}
\frac {\partial J} {\partial \mathbf v_c} & = \frac \partial {\partial \mathbf v_c} -\log P(O=o|C=c) \\
& = \frac \partial {\partial \mathbf v_c} \left ( - \mathbf u_0^T \mathbf v_c +\log \sum_{w\in \text{Vocab}} \exp {\mathbf u}_w^T \mathbf v_c \right ) \\
& = - \mathbf u_0 + \frac { \sum_{w \in \text{Vocab}} \exp \left (\mathbf u_w^T \mathbf v_c \right ) \cdot \mathbf u_w } { \sum_{w\in \text{Vocab}} \exp \mathbf u_w^T \mathbf v_c } \\
& = -\mathbf u_0 +\sum_{w\in\text{Vocab}} P(O=w|C=c) \cdot \mathbf u_w \\
& = - U^T \mathbf y + U^T \hat {\mathbf y} \\ 
& = U^T (\mathbf {\hat y} - \mathbf y) 
\end{split}
\]
$U^T \mathbf {\hat y}$ has $[D, 1]$ shape because $U^T$ has $[D, W]$ shape, and $\mathbf {\hat y}$ has $[W, 1]$ shape.


\item \,
\begin{enumerate}
\item If $w=o$ then, 
\[
\begin{split}
\frac {\partial J} {\partial \mathbf u_o} &= \frac \partial {\partial \mathbf u_o} -\log P(O=o|C=c) \\
&= \frac \partial {\partial \mathbf u_o} \left ( - \mathbf u_0^T \mathbf v_c +\log \sum_{w\in \text{Vocab}} \exp {\mathbf u}_w^T \mathbf v_c \right ) \\
&= - \mathbf v_c + \frac { \exp \left ( \mathbf u_o^T \mathbf v_c \right ) \mathbf v_c} { \sum_{w \in \text{Vocab}} \exp \left (\mathbf u_w^T \mathbf v_c \right ) \cdot \mathbf v_c }  \\
&= (P(O=o|C=c)-1)\mathbf v_c \\
& = (\hat y_o -1) \mathbf v_c
\end{split}
\]

\item If $w\ne o$ then,
\[
\begin{split}
\frac {\partial J} {\partial \mathbf u_w} &= \frac \partial {\partial \mathbf u_w} -\log P(O=o|C=c) \\
&= \frac \partial {\partial \mathbf u_w} \left ( - \mathbf u_0^T \mathbf v_c +\log \sum_{w'\in \text{Vocab}} \exp {\mathbf u}_{w'}^T \mathbf v_c \right ) \\
&= \frac { \exp \left ( \mathbf u_w^T \mathbf v_c \right ) \mathbf v_c} { \sum_{w' \in \text{Vocab}} \exp \left (\mathbf u_{w'}^T \mathbf v_c \right ) \cdot \mathbf v_c }  \\
&= P(O=w|C=c)\mathbf v_c \\
& = \hat y_w \mathbf v_c
\end{split}
\]
\end{enumerate}


\item Noted: We haved defined the length of Vocabulary as $W$.
\[
\frac {\partial J} {\partial \mathbf U} = 
\begin{bmatrix} 
\frac {\partial J} {\partial \mathbf u_1} \\
\vdots \\
\frac {\partial J} {\partial \mathbf u_o} \\
\vdots \\
\frac {\partial J} {\partial \mathbf u_W} \\
\end{bmatrix} =
\begin{bmatrix} 
\hat y_1 \mathbf v_c^T  \\
\vdots  \\
(\hat y_0 -1) \mathbf v_c^T \\
\vdots \\
\hat y_W \mathbf v_c^T 
\end{bmatrix} =
\begin{bmatrix}
\hat y_1 \\
\vdots \\
\hat y_o -1 \\
\vdots \\
\hat y_W 
\end{bmatrix} \mathbf v_c^T
\]

\item Differentiate the sigmoid function. 
\[
\frac {d\sigma (x)} {dx} = \frac d {dx} \frac 1 {1 + e^{-x}} = \frac {e^{-x}} {(1+e^{-x})^2} = \frac 1 {1+e^{-x}} \frac {e^{-x}} {1+e^{-x}} = \sigma(x)(1-\sigma(x))
\]


\item \,
\begin{enumerate}
\item Repeat (2). Differentiate $J$ with respect to $\mathbf v_c$ 
\[
\begin{split}
\frac {\partial} {\partial \mathbf v_c} J  & = -\frac {\partial} {\partial \mathbf v_c} \log (\sigma (\mathbf u_o^T \mathbf v_c)) - \sum_{k=1}^K \frac {\partial} {\partial \mathbf v_c} \log (\sigma(-\mathbf u_k^T \mathbf v_c)) \\
& = -(1-\sigma(\mathbf u_o^T \mathbf v_c)) \mathbf u_o - \sum_{k=1}^K (1-\sigma(-\mathbf u_k^T \mathbf v_c)) (-\mathbf u_k) \\
& = (\sigma(\mathbf u_o^T \mathbf v_c) -1)\mathbf u_o + \sum_{k=1}^K (1-\sigma(-\mathbf u_k^T \mathbf v_c)) \mathbf u_k
\end{split}
\]

\item Repeat (3). Differentiate $J$ with respect to $\mathbf u_o$ 
\[
\begin{split}
\frac \partial {\partial \mathbf u_0} J & = - \frac \partial {\partial \mathbf u_0} \log (\sigma (\mathbf u_o^T \mathbf v_c)) \\
&= - \frac 1 {\sigma(\mathbf u_o^T \mathbf v_c)} \frac \partial {\partial \mathbf u_0} \sigma (\mathbf u_o^T \mathbf v_c) \\
& = - \frac 1 {\sigma(\mathbf u_o^T \mathbf v_c)} \sigma(\mathbf u_o^T \mathbf v_c)(1- \sigma(\mathbf u_o^T \mathbf v_c)) \frac \partial {\partial \mathbf u_0} \mathbf u_o^T \mathbf v_c \\
& = (\sigma(\mathbf u_o^T \mathbf v_c)-1) \mathbf v_c
\end{split}
\]
\item Repeat (3). Differentiate $J$ with respect to $\mathbf u_k$ 
\[
\begin{split} 
\frac \partial {\partial \mathbf u_k} J &= -\frac \partial {\partial \mathbf u_k} \log(\sigma (-\mathbf u_k^T \mathbf v_c)) \\
& = (1-\sigma(-\mathbf u_k^T \mathbf v_c)) \mathbf v_c
\end{split}
\]
\end{enumerate}

The reason is that it takes $O(W^2)$ times to calculate $\mathbf {\hat y}, \mathbf y$. Therefore, it takes quadratic time to compute the native-softmax loss. On the other hand, it takes $O(k)$ times to compute the Negative Sampling loss. 

\item Repeat the previous exercise without the distinct sampling assumption. As you can see, calculating the derivative with respect to $\mathbf v_c, \mathbf u_o$ does not use the assumption. Therefore, these derivatives are the same as the previous ones. 
\[ 
\frac \partial {\partial \mathbf u_k} J = -\sum_{w_k = w_{k'}} \log(\sigma(-\mathbf u_k^T \mathbf v_c)) = (1-\sigma(-\mathbf u_k^T \mathbf v_c)) \mathbf v_c \times\sum_{k'=1}^K [w_k = w_{k'} ]
\]
, where $[ \text{true} ] = 1, [ \text{false} ] = 0$.


\item \,

\[
\begin{split}
\frac {\partial J_\text{skip-gram} (\mathbf v_c, w_{t-m}, \dots, w_{t+m}, \mathbf U) } {\partial \mathbf U} &= \sum_{-m \le j \le m, j\ne 0} \frac {\partial J(\mathbf v_c, w_{t+j}, \mathbf U)} {\partial \mathbf U} \\
\frac {\partial J_\text{skip-gram} (\mathbf v_c, w_{t-m}, \dots, w_{t+m}, \mathbf U) } {\partial \mathbf v_c} &= \sum_{-m \le j \le m, j\ne 0} \frac {\partial J(\mathbf v_c, w_{t+j}, \mathbf U)} {\partial \mathbf v_c} \\
\frac {\partial J_\text{skip-gram} (\mathbf v_c, w_{t-m}, \dots, w_{t+m}, \mathbf U) } {\partial \mathbf v_w} &= 0
\end{split}
\]

\end{enumerate}
\end{document}




















